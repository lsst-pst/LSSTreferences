\documentclass[12pt]{article}
\usepackage{amsmath}
\usepackage{color}
\usepackage[colorlinks=true, citecolor=blue, linktocpage=true]{hyperref} % n.b. glossary colour is set in style
%
\usepackage[titletoc,title]{appendix}
%\usepackage[style=authoryear, uniquename=false, backend=biber]{biblatex}
\usepackage[style=authoryear, uniquename=false, backend=bibtex]{biblatex}
\usepackage{graphicx}
\usepackage{import}
\usepackage{journals}

\bibliography{LSSTbiblio}
%
\begin{document}
\title{Citations for LSST papers}
\author{The LSST Project Science Team}
%\date{1996-03-06}
\maketitle


\begin{abstract}
This document provides information about science and technology papers that describe LSST infrastructure.
These papers should be referenced when describing the LSST system and its anticipated science
outcome. Doing so will refer the readers to the most relevant publications and also recognize the 
contributions of those who brought the Project to fruition.
\end{abstract}


\tableofcontents

%------------------------------------------------------------------------------
\newpage
\section{LSST Project Publication Policy} 

The LSST Project Publication Policy can be obtained from the LSST website.
The remainder of this document lists suggested papers to reference, organized by topics. 

Files needed to make this file are available from: \\
https://github.com/lsst-pst/LSSTreferences

\section{LSST System and Science}

The LSST system (brief overview of telescope, camera and data management subsystems),
science drivers and science forecasts are described in:

\begin{itemize}
\item LSST Science Requirements Document: \cite{lsstSRD}
\item LSST overview paper: \cite{ivezic2008lsst} 
\item LSST Science Book: \cite{abell2009lsst}
\end{itemize}
%------------------------------------------------------------------------------


\section{Simulations}

The LSST simulations are described in a series of papers. Use of the LSST simulations should cite the LSST simulations overview paper \cite{2014SPIE.9150E..14C} and the specific simulation tools used:

\begin{itemize}
\item LSST Catalogs (CatSim): \cite{2014SPIE.9150E..14C}
\item Operations Simulator (OpSim): \cite{2014SPIE.9150E..15D}
\item Metrics Analysis Framework (MAF): \cite{2014SPIE.9149E..0BJ}
\item Image simulations (Phosim): \cite{0067-0049-218-1-14}
\end{itemize}
%------------------------------------------------------------------------------


\section{Data Management}

LSST data management system and the data products are described in: 

\begin{itemize}
  \item The LSST Data Management System: \cite{2015arXiv151207914J}
  \item Data Products Definition Document: \cite{DPDD}
\end{itemize}
 %------------------------------------------------------------------------------


\section{Camera}

\begin{itemize}
   \item Design and development of the LSST camera: \cite{2010SPIE.7735E..0JK}
\end{itemize}
%------------------------------------------------------------------------------


\section{Telescope and Site}

\begin{itemize}
   \item Telescope and site overview and status in 2014:  \cite{2014SPIE.9145E..1AG}
   \item LSST OCS Scheduler design: \cite{2014SPIE.9149E..0GD}
\end{itemize}
%------------------------------------------------------------------------------

\section{System Engineering}

\begin{itemize}
   \item LSST systems engineering: \cite{2014SPIE.9150E..0MC}
   \item System verification and validation: \cite{2014SPIE.9150E..0NS}
\end{itemize}
%


%\nocite{*}                          % * includes all non-cited entries in biblio.bib

\printbibliography[heading=bibintoc]

\end{document}
